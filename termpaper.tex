% Term paper proposal template - Ilona Sparks
% CSC 300: Professional Responsibilities
% Dr. Clark Turner

% One Column Format
\documentclass[11pt]{article}

\usepackage{hyperref}
\usepackage{multicol}
\usepackage{setspace}
\usepackage{url}

\hypersetup{
    bookmarks=true,         % show bookmarks bar?
    colorlinks=true,        % false: boxed links; true: colored links
    linkcolor=black,        % color of internal links
    citecolor=black,        % color of links to bibliography
    filecolor=black,        % color of file links
    urlcolor=black          % color of external links
}

%%% PAGE DIMENSIONS
\usepackage{geometry} % to change the page dimensions
\geometry{letterpaper}


\begin{document}

\title{\vfill Coffee in Iceland: The Ethics of Java on Android} %\vfill gives us the black space at the top of the page
\author{
By Ross Light \vspace{10pt} \\
CSC 300: Professional Responsibilities  \vspace{10pt} \\
Dr. Clark Turner \vspace{10pt} \\
}
\date{\today} %Or use \Today for today's Date

\maketitle

\vfill  %in combination with \newpage this forces the abstract to the bottom of the page
\begin{abstract} % {{{1

Google released Android on November 5, 2007 \cite{open-handset-alliance-ann} as
a platform for mobile phones. Programmers can write applications for Android in
Java.  These programs are run with Dalvik --- a virtual machine that Google
created.  In 2010, Oracle~America~Inc.\ filed a lawsuit against Google
\cite{oracle-lawsuit}, claiming that Dalvik infringed on Oracle's intellectual
property.  This raises the question, is Google's use of the Dalvik virtual
machine in Android ethical?  Creating Dalvik was the only way to ethically meet
performance, memory, and licensing requirements, which aligns with the spirit of
the ACM Software Engineering Code of Ethics and Professional Practice
\cite{secode}.

(It should be noted that the author of this paper is employed by Google at time
of writing.)

\end{abstract} % }}}1

\thispagestyle{empty} %remove page number from title page, but still keep it as pg #1
\newpage

%Create a table of contents with all headings of level 3 and above.
\thispagestyle{empty}  %Remove page number from TOC
\tableofcontents

\newpage

%start 2 column format
\begin{multicols}{2}
%Start numbering first page of content as page 1
\setcounter{page}{1}

\section{Facts} % {{{1
\label{sec:facts}

In late 2007, Google (along with a handful of other companies) announced the
Open Handset Alliance \cite{open-handset-alliance-ann} along with the operating
system it wished to promote: Android.  The Android operating system is built
entirely from open source projects.  Most of the projects Android uses are
released under the Apache License 2.0, with the exception of the Linux kernel
(which is released under the GNU GPL).  Google provides a software development
kit (SDK) that allows developers to write applications for Android in the Java
programming language.

Java is a programming language created by James Gosling that was originally
maintained by Sun Microsystems.  In 2006, Sun announced Java's release under the
open source GNU GPL license. \cite{sun-open-sources-java} On August 20, 2009,
the U.S.~Justice Department approved Oracle's deal to acquire Sun Microsystems
and its intellectual property, including Java. \cite{oracle-buys-sun}

Many programming languages are converted from text directly into instructions
that the CPU understands.  Java does not get converted directly to CPU
instructions --- in order to run Java code, one must use a virtual machine.  A
virtual machine guarantees that the same program will run on any CPU without
modification.  Once a virtual machine has been written for a particular CPU
architecture, it can execute programs written for the virtual machine.

The Java compiler produces a documented, well-defined set of standard bytecodes,
known as the JavaVM specification. \cite[Chapter 4]{javavm-bytecode}  There are
many different implementations of the JavaVM specification: HotSpot
(Sun/Oracle's de facto implementation) \cite{hotspot}, Apache Harmony
\cite{apache-harmony}, and Red Hat's IcedTea \cite{icedtea}, to name a few.

The developers of Android created a new virtual machine --- Dalvik.  Dalvik
executes bytecodes in a format that is different from JavaVM in order to fulfill
the performance requirements on mobile phones.  \cite{dalvik-bytecode}
\cite{dalvik-vm-arch} The Android~SDK includes a program, dx, that converts
Java bytecodes produced by Oracle's Java compiler into Dalvik bytecodes.
\cite{android-sdk-building} In order to provide compatibility with the Java
libraries, Dalvik includes portions of the Apache~Harmony library.
\cite{apache-harmony} \cite{dalvik-readme}  Apache~Harmony and Dalvik are
released under the Apache License 2.0 \cite{apache-license} which is ``preferred
by many companies because such licenses make it possible to use open-source
software code without having to turn proprietary enhancements back over to the
open source software community.'' \cite{why-apache2-license}

On August 12, 2010, Oracle filed a lawsuit against Google, claiming that Dalvik
infringes upon their intellectual property. \cite[Count VIII]{oracle-lawsuit}

% end facts }}}1

\section{Research Question} % {{{1
\label{sec:question}

Is Google's use of the Dalvik virtual machine in Android ethical?

If Google is unethical, then their partners in the Open Handset Alliance are
unethical.  The Android platform is used on a wide variety of phones currently
on the market distributed by various different companies.  All of these
companies would be using unethically obtained software.  This would be
disastrous for many companies and would harm the public.  Furthermore, the
question also has implications beyond Android.  If Google is unethical in using
Dalvik, other creators of compatible virtual machines could be just as
unethical.  This would include Red Hat and the Apache Software Foundation ---
two high-profile open~source developers.

% end question }}}1

\section{Extant Arguments} % {{{1
\label{sec:args}

\subsection{Oracle's Claim} % {{{2
\label{sub:args_oracle}

Oracle's claim is that ``as a direct and proximate result of Google's direct
and indirect willful copyright infringement, Oracle America has suffered, and
will continue to suffer, monetary loss to its business, reputation, and
goodwill.'' \cite[p.~9 line 11]{oracle-lawsuit}

The first claim being made is that Google is infringing on Oracle's copyright.
The second claim is that by infringing, Google is harming the public by causing
losses to Oracle's employees and investors.  Either of these actions could be
unethical, and the first would certainly be illegal.

% sub args_oracle }}}2

\subsection{Groklaw's Response} % {{{2
\label{sub:args_groklaw}

When the lawsuit was announced, the website Groklaw posted an analysis of the
lawsuit.  Groklaw is a website run by Pamela Jones, an open source advocate who
previously worked as a paralegal. \cite{groklaw-pj} Jones responded to Oracle's
lawsuit, saying: \cite{groklaw}

\begin{quotation}
I expect Google would say Dalvik was an alternative to Java, not a version of
it. If indeed none of the Sun employees that ended up at Google worked on this,
and it's built from the ground up without any Sun technology or [Intellectual
Property], on what basis can Oracle prevail? Perhaps Oracle figures no room at
Google is clean enough. And of course clean room means nothing when it comes to
patents.
\end{quotation}

Jones's argument is that if Google has not used any of Oracle's software to
build Dalvik, then there's no violation of intellectual property, and thus no
unethical behavior.

% sub args_groklaw }}}2

\subsection{The Copied Code Debate} % {{{2
\label{sub:fosspatents}

In January 2011, Engadget (a popular technology news website), posted an article
indicating that Google may be copying portions of Oracle's code in the Android
source code repository.  \cite{android-copies-java-code}  Florian Mueller, the
author of the blog ``FOSS Patents'', originally discovered the similarities in
the files.  \cite{fosspatents} These observers claim that Google is unethical
because Google directly copied source code from Oracle.

% TODO: Cite Exhibit J

Oracle also used a similar example in their lawsuit as proof that Google was
aware of the copying.  Exhibit J shows a side-by-side comparison of Oracle's
source code to Google's source code.  If any portion of Oracle's software is
copied, it would be unauthorized duplication, which is illegal and unethical.

Ars Technica (another technology news website) published an article soon after
claiming that SONiVOX, a third-party company, had accidentally committed the
offending changes.  \cite{ars-tech-copying}  The article made the case that
because Google was not aware of the code's presence and they swiftly removed the
offending code, they were still ethical.

% subsection fosspatents }}}2

\subsection{Summary of Arguments} % {{{2
\label{sub:args-summary}

The main criticism of Dalvik is its use of Java.  Because Java is Oracle's
product, it infringes on their intellectual property and should not be used.
Critics claim that Dalvik copies code from Oracle and is using it unlawfully and
unethically.  Therefore, Google is harming Oracle and its shareholders by
engaging in unfair business practices, effectively using Oracle's products
against Oracle.

Supporters of Dalvik claim that Dalvik is an alternative and fulfills the same
purpose as Oracle's HotSpot virtual machine, but does not use any of the
HotSpot's software, it is ethical.  SONiVOX's code duplication was unethical,
but Google was not aware of its use and it was removed immediately upon
discovery.  The files used were only unit tests and not shipped as part of
Android.

% subsection args-summary }}}2

% end args }}}1

\section{Analysis} % {{{1
\label{sec:analysis}

\subsection{Criteria for Analysis} % {{{2
\label{sub:criteria}

% TODO
Much of the ethics of the situation is concerned with Oracle's intellectual
property.  The scope of the question is much wider, though.  Whether Dalvik is
ethical is not just dependent on whether it just violates principles, but also
whether it follows principles.

In order to establish whether Google's actions are ethical or unethical, it is
important to provide an objective set of standards that are well-accepted for
the basis of this analysis.  At first glance, using Google's Code of Conduct
\cite{google-conduct} seems like a reasonable choice, and many parts of Google's
Code of Conduct are applicable here.  However, this analysis will use the ACM
Software Engineering Code of Ethics and Professional Practice \cite{secode}, as
it is non-partial and includes more restrictions than Google's Code of Conduct
does.

Google is primarily a software company, consisting of software engineers that
create products, including Dalvik.  The ACM Software Engineering Code of Ethics
and Professional Practice ``prescribes [the Code] as obligations of anyone
claiming to be or aspiring to be a software engineer.'' \cite{secode} By this
measure, Google would be unethical in creating Dalvik if it violated the
Software Engineering Code of Ethics and Professional Practice \cite{secode} or
if there was a higher principle that it conflicts with.

% subsection criteria }}}2

\subsection{The Use of Java} % {{{2
\label{sub:java}

The Android SDK provides a modified Eclipse environment for developing Android
applications using the Java programming language.  In light of the lawsuit,
using another company's programming language seems unethical because it violates
section 2.02 of the Software Engineering Code of Ethics (``Not knowingly use
software that is obtained or retained either illegally or unethically'' \cite[\S
2.02]{secode}).  Furthermore, using Java seems like a particularly bad idea when
Google could have simply avoided the issue by picking another language to use.
I address the licensing issues regarding Java in section~\ref{sub:licensing},
``Licensing'', but first, why did Google pick Java?

Java refers to quite a few different concepts, so it is important to remember
that there are two components that Java encompasses: the compiler and the
virtual machine.  Oracle provides both a compiler and a virtual machine.  The
compiler takes the source code the programmer writes and converts into
bytecodes, which is then run by the virtual machine.

Java is a mature language that is taught at many universities, including Cal
Poly, so its use reduces the amount of work for a developer to create an Android
application.  A developer doesn't have to learn a new language in order to be
productive.

% TODO: fix TIOBE cite
The Software Engineering Code of Ethics presents an obligation to
``work to follow professional standards, when available, that are most
appropriate for the task at hand, departing from these only when ethically or
technically justified.'' \cite[\S 3.06]{secode}  Java was ranked the most
popular programming language in 2006 \cite[Long term trends]{tiobe}, the year
before Android was released.  Java runs on many platforms and is documented in
specifications.  Java is a common standard used by many other software
companies.  As long as there isn't an ethical or technical conflict, the
Software Engineering Code of Ethics would recommend Java (the professional
standard) in this case.

\subsubsection{The dx Program} % {{{3
\label{ssub:dex}

In an ordinary Java project, a developer compiles his Java source code files and
those produce JavaVM bytecodes.  Those bytecodes can be run directly by
a Java virtual machine.  An Android application requires an extra step: dx.
dx, a part of the Android SDK, is a program that converts Java bytecodes to
Dalvik bytecodes (the Dalvik executable format, dex).  Dalvik doesn't understand JavaVM
bytecodes at all; the architecture is fundamentally different.

Because Dalvik uses dex, Java bytecodes are an intermediary, not the final
product.  Converters could be written that compile other programming languages
to the Dalvik executable format (and projects already exist to do this).  This
means that Dalvik exists completely independently of Java.  Dalvik is a generic
virtual machine.

Dalvik, then, must be categorized differently than how Oracle is portraying it.
Dalvik is not an implementation of Java, it is an alternative, just as Pamela
Jones argued. \cite{groklaw}  Oracle's HotSpot and Google's Dalvik solve the
same problem, but they do it in different ways.  Copying someone else's
implementation is certainly intellectual property infringement, but making a
different program that accomplishes the same task is ethical.  For instance, no
one is arguing that Google Docs is violating Oracle's OpenOffice intellectual
property, even though they share many of the same features.  Products can solve
the same problem and still be ethical.

% subsubsection dex }}}3

\subsubsection{Damages to Oracle} % {{{3
\label{ssub:oracle-damage}

Oracle's lawsuit would seem to indicate that Google is harming the public
(specifically Oracle) by using unfair business practices to detract from
Oracle's profits.  Oracle publishes a specification for a similar environment,
Java Micro Edition, but there are no well-known phones that implement it.
Before Android's announcement in 2007, the usage of Java was declining, but 2008
showed a sharp increase in interest.  \cite{tiobe}  Android may not be directly
related to this increase, but with an influx of developers, it would follow that
developers will be looking for documentation and resources related to Java.
These developers may continue to use Java for other software projects to
minimize confusion.

The Java compiler is already freely distributed under GNU GPL.  Oracle makes
their money in services relating to Java, not Java itself.  The lack of phone
supporting Java Micro Edition suggests that Oracle is not too heavily invested
in the technology.  Nicholas Artman, a former employee at Apple and Yahoo,
summarized that ``Oracle is, in essence, suing Google for actively preventing
Oracle from hemoraging money on phones.'' \cite{artman}

% subsubsection oracle-damage }}}3

\subsubsection{Summary} % {{{3
\label{ssub:java-summary}

Java is now a well-established programming language in software engineering.
The Software Engineering Code of Ethics encourages the use of standards whenever
possible, so using Java is the proper choice unless there is a conflicting
ethical or technical reason.  By separating Dalvik from Oracle's Java
implementation, the Android SDK can use the Java programming language without
unethically using Oracle's software.  Including the Java programming language
promotes Oracle's products and helps them, rather than hindering them.  The
choice to use Java is ethical.

% subsubsection java-summary }}}3

% subsection java }}}2

\subsection{Licensing} % {{{2
\label{sub:licensing}

Section 2.02 of the Software Engineering Code states that software engineers
should ``not knowingly use software that is obtained or retained either
illegally or unethically.'' \cite[\S 2.02]{secode}  Oracle claims that Google is
doing exactly that: using software that infringes on their intellectual
property.  To evaluate this claim, we need to examine the licensing terms on
Oracle's Java implementation and Dalvik.

Oracle releases their implementation of the compiler and the virtual machine
under the GNU~General~Public~License, which means that it can be redistributed
or modified as long as the source code is distributed as well.  This is a common
open source license, but it is referred to as a viral license, because any
modifications have to be made open as well.  One of the requirements of Android
is to provide a common platform for a wide variety of phones
\cite{open-handset-alliance-ann}; the only way to ensure a common platform is to
encourage widespread adoption.  Companies want to be able to make proprietary
modifications for their implementations, the GNU~General~Public~License does not
allow this.  The Apache~License~2.0, another common open source license, allows
any modification or redistribution, as long as the original copyright notice
remains intact.  Companies prefer the Apache~License~2.0 because they want to be
able to make proprietary modifications.  \cite{why-apache2-license}  Thus, to
promote adoption, Google needed to release Android (and Dalvik) under the
Apache~License~2.0 (or something similar).  However, code released under the
GNU~General~Public~License cannot legally be released under the
Apache~License~2.0.

Google does not include any Java development tools in the Android SDK; the Java
Development Kit must be installed separately from Oracle.  This means that
Google is not redistributing or modifying the Java compiler.  A major part of a
Java virtual machine is the runtime libraries, which provides code that is
common to many software projects.  Oracle does include an extensive standard
library in their Java Development Kit, but the Android SDK specifically disables
its use when it compiles an application.  Instead, Google uses the
Apache~Harmony libraries, an implementation of a subset of Oracle's standard
library released under the Apache~License~2.0.  In order for this to be ethical,
the Software Engineering Code of Ethics dictates that software engineers ought
to ``credit fully the work of others and refrain from taking undue credit.''
\cite[\S 7.03]{secode}  Google distributes Dalvik under the same
Apache~License~2.0 and credits Harmony in the Dalvik README.
\cite{dalvik-readme}  The principle of giving credit is upheld.

\subsubsection{The Unit Test Fiasco} % {{{3
\label{ssub:unittest-fiasco}

As noted before, Florian Mueller made the claim on January 2011 that files in
the Dalvik libraries were copied directly from Oracle's HotSpot.  It was later
shown that these files were only unit tests, which makes the ethics of the
situation a little fuzzy.  It has also been revealed that the files were
committed by a third-party company, SONiVOX, involved in the project.  Google
did not know that they had copied the code, and once they discovered the code
was copied, they deleted the offending files from the repository.  Although they
used software that was retained illegally, they did not know about it.  Once
they did discover it, they deleted the files.  Therefore, they acted in
accordance with Section 2.02 of the Software Engineering Code of Ethics.

% subsubsection unittest-fiasco }}}3

\subsubsection{Summary} % {{{3
\label{ssub:licensing-summary}

Instead of using Oracle's libraries, Google used the Apache Harmony libraries.
Dalvik uses its own executable format that is distinct from JavaVM.  Google
created Dalvik and then used Apache~Harmony to supplement its functionality,
obeying all the terms of the Apache~License~2.0.  All software was either
created at Google or obtained legally.

% subsubsection licensing-summary }}}3

% subsection licensing }}}2

\subsection{Duty to the Public} % {{{2
\label{sub:public}

As part of a software engineer's duty to the public, a software engineer should
``approve software only if they have a well-founded belief that it is safe,
meets specifications, passes appropriate tests, and does not diminish quality of
life, diminish privacy or harm the environment.'' \cite[\S~1.03]{secode}  The
possibility of its damage is limited because Android operates on consumer-level
mobile devices.  No claim has been made that Android is not well-tested and it
comes from the Linux kernel, a well-tested piece of software.  Therefore, the
only relevant concerns here are the conformance to specifications, the effect on
quality of life, and the effect on privacy.  Dalvik must meet all of these
conditions in order to be ethical.

\subsubsection{Requirements} % {{{3
\label{ssub:requirements}

The specification of a software product includes the requirements (goals and
restrictions) and the details of how to meet those requirements (design
documents and user documentation).  If Dalvik does not meet its requirements,
then it does not follow its specification, and therefore it is unethical.

When the Open Handset Alliance announced Android, they cited three major
design goals for Android.  The main requirements are: \cite{open-handset-alliance-ann}

\begin{itemize}
    \item User Experience
    \item Openness
    \item Portability
\end{itemize}

Promoting a good user experience by definition is improving quality of life for
the user, so that is an ethical goal.  An open operating system is (arguably)
neutral from the user's point of view, but it allows companies to share code:
Motorola can make bug fixes that assist Samsung, and vice versa, which
contributes to the user experience (i.e. less unexpected behavior).  Portability
means that the same software, which is well-tested and mature, can be used on
many phones and still be useful.  These requirements are worthy and ethical, but let's
see how they apply to the Software Engineering Code of Ethics.

% subsubsection requirements }}}3

\subsubsection{Why a Virtual Machine?} % {{{3
\label{ssub:why-vm}

In considering these requirements, none of these seems to mandate the use of a
virtual machine.  Indeed, Apple's competing iOS compiles directly to machine
code, runs on devices with different ARM processors, and provides a good user
experience.  The open licensing requirement can be fulfilled in any way.  Why
was Dalvik chosen as the solution for the requirements?

One of Android's goals is to provide portability: an application should be able
to run on a variety of devices.  Each phone may have a slightly different
hardware; a program which runs well on one phone may run poorly on another
phone.  A virtual machine would provide a predictable, consistent execution
environment between phones without placing burden on application developers.

A virtual machine also has some side benefits that are useful for phone
development.  Because a virtual machine has more control over the execution
environment, a virtual machine can more aggressively collect unused memory.  The
virtual machine can also track all accesses to resources, so a virtual machine
can enforce that a program only has access to the APIs it requested permission
for (this is discussed in more detail later).  Dalvik uses these features to
make Android secure and memory efficient --- important qualities for maintaining
user privacy and user experience.

% subsubsection why-vm }}}3

\subsubsection{Quality of Life} % {{{3
\label{ssub:quality}

The first goal of the Android platform is to provide a good user experience.
This encourages the quality of life aspect of the Software Engineering Code of
Ethics by making the user's life better by using the phone.  Also, because
Android is portable and runs on many phones, users don't have to learn a new
interface every time they get a new phone (which is a rather frequent occurrence
in the United States).  Although the primary user is the general public,
application developers are also a large part of the audience of Android.  An
application can be written once and deployed on many different phones with
radically different hardware specifications.  When the first Android tablets
were introduced, most applications ran correctly on the first day because the
virtual machine could handle the better processor and would abstract the
hardware.  The end-user benefits from this abstraction as well.  Applications
behave nicely on their phone: resources are managed, the process is secure, and
the application is guaranteed to run.

% subsubsection quality }}}3

\subsubsection{Privacy and Security} % {{{3
\label{ssub:privacy}

The Software Engineering Code of Ethics dictates that software should not
adversely affect users' privacy.  Dalvik doesn't directly work with private
data, but it runs other programs that do.  In this case, Dalvik should promote
good practices that aide privacy.  This is Dalvik's security layer.

Mobile phones are a significant part of people's lives and contains personal
information such as contacts and browsing history.  With a proprietary phone
operating system, it is difficult for security experts and privacy enthusiasts
to see how their data is being used.  Providing an open source platform for
mobile phones encourages user privacy by allowing external audits to ensure that
the operating system is not transmitting information without consent.

Whenever a user installs a new application from the Android Market, he or she is
clearly presented with the list of permissions that application is requesting.
This is part of Dalvik's runtime environment: applications must declare what
actions they are going to try to do, or else Dalvik terminates the program.
Dalvik controls access to private information.

This would be difficult (although not impossible) to do without a virtual
machine.  Natively compiled code uses the CPU directly, so there would be more
burden on the underlying operating system to implement the security features,
which are usually rather abstract.  For example (taken from the Android
Developer Documentation), a picture viewing program can't ordinarily access data
from the email client, but if the user wants to open an attachment, the email
client can temporarily grant access to the corresponding image, and nothing
more.  By controlling the code executing in user space, Dalvik can enforce much
more fine-grained control over security, thus protecting its user's privacy.

% subsubsection privacy }}}3

\subsubsection{Effect on the Public} % {{{3
\label{ssub:public-effect}

Two of the duties the Software Engineering Code of Ethics identifies are the
users and the public.  Since the primary user of Android is the public, this is
a significant duty for Google.  The requirements (and the product) should
provide utility to the public in order to be ethical.

This idea of not diminishing quality of life relates to the ethical theory of
utilitarianism: that which maximizes utility (or benefit) over a group of people
is what is good.  It provides a (more or less) objective basis for ethics, and
it is a higher principle that also applies outside of software engineering.  To
be consistent with utilitarianism, Android must provide a net positive (or
neutral) effect on the total utility.  Taking away utility would be diminishing
quality of life.

One of Android's main design goals is user experience: ensuring that a user can
get tasks done in Android effectively and easily.  That user experience is
consistent across many phones (part of the portability), which will make the
transition to another phone smoother.  This shortens the amount of collective
time that users have to learn another user interface paradigm, so that people
can get things done.  This is an important benefit.

% subsubsection public-effect }}}3

\subsubsection{Summary} % {{{3
\label{ssub:public-summary}

An important part of the Software Engineering Code of Ethics is ensuring that
software contributes to the public good.  Android's requirements are designed to
help the public.  Using Dalvik promotes all of the project's requirements.
Dalvik increases security, which keeps users in control of what data each
program is allowed to access.  Because of this, Dalvik fulfills Section 1.03 of
the Software Engineering Code of Ethics.

% subsubsection public-summary }}}3

% subsection public }}}2

% section analysis }}}1

\section{Conclusion} % {{{1
\label{sec:conclusion}

In order to best meet project requirements, Google needed to use Java and a
virtual machine in Android.  Google fulfilled their duty to the public and the
Software Engineering Code of Ethics by creating a new virtual machine, Dalvik.
Because Dalvik uses a different architecture and doesn't use code from Oracle's
Java implementation, it does not infringe upon Oracle's intellectual property
and remains ethical.  Dalvik meets the requirements of the Android project by
promoting user experience, portability, security, and privacy.  These
requirements could not be met by existing software, so creating Dalvik was the
logical choice.  Google's use of Dalvik is ethical.

% section conclusion }}}1

\appendix

\section{Glossary} % {{{1
\label{sec:glossary}

\begin{description}
    \item[Virtual Machine] A program that runs instructions that are not
    natively understood by a CPU.
    \item[Bytecode] An instruction understood by a virtual machine.
    \item[JavaVM] The set of bytecodes that is documented by ``The Java Virtual
    Machine Specification''. \cite{javavm-bytecode}
    \item[SDK] Software Development Kit.  An SDK is a set of programs and other
    materials that allows programmers to write programs for a particular
    environment.
    \item[Open Source] If a software project released as open source, it means
    that the software's source code can be obtained and reused legally.  There
    are many licenses that are considered open source, each with different
    terms.
    \item[Apache License 2.0] An open source license that allows modification
    and reuse of the source code, even if the user don't redistribute the
    modified source code. \cite{apache-license}
    \item[GNU GPL License] GNU General Public License.  An open source license
    that allows modification and reuse of the source code, as long as users
    distribute the source code of their modifications. \cite{gpl-license}
    \item[Library] A collection of code that is useful in many different
    programs (e.g.\ text handling, graphics, etc.).
\end{description}

% section glossary }}}1

\end{multicols}
\newpage

%cite all the references from the bibtex you haven't explicitly cited
\nocite{*}
\bibliographystyle{IEEEannot}
\bibliography{termpaper}

\end{document}

% vim: ft=tex tw=80 fdm=marker fdc=4 spell

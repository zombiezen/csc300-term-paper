% Term paper proposal template - Ilona Sparks
% CSC 300: Professional Responsibilities
% Dr. Clark Turner

% One Column Format
\documentclass[12pt]{article}

\usepackage{setspace}
\usepackage{url}

%%% PAGE DIMENSIONS
\usepackage{geometry} % to change the page dimensions
\geometry{letterpaper}


\begin{document}

\title{\vfill Coffee in Iceland: The Ethics of Java on Android} %\vfill gives us the black space at the top of the page
\author{
By Ross Light \vspace{10pt} \\
CSC 300: Professional Responsibilities  \vspace{10pt} \\
Dr. Clark Turner \vspace{10pt} \\
}
\date{\today} %Or use \Today for today's Date

\maketitle

\vfill  %in combination with \newpage this forces the abstract to the bottom of the page
\begin{abstract}
Google's Android platform was released on November 5, 2007
\cite{open-handset-alliance-ann} as an open-source operating system for
consumer-level mobile devices, especially phones.  Android is based on Linux,
and exposes an application program interface (API) for the Java programming
language.  To improve performance, Android uses a custom virtual machine ---
called Dalvik --- to run the Java programs deployed on Android.  Three years
after Android's release, Oracle America Inc.~filed a lawsuit against Google
\cite{oracle-lawsuit}, claiming that Dalvik infringed on Oracle's registered
patents.  Google's use of the Dalvik virtual machine in Android is ethical
because it does not violate any portion of the Software Engineering Code of
Ethics and Professional Practice \cite{secode} and it is a product that was
created to meet requirements ethically.

(It should be noted that the author of this paper is employed by Google at
time of writing.)

\end{abstract}

\thispagestyle{empty} %remove page number from title page, but still keep it as pg #1
\newpage

%%%%%%%%%%%%%%%%%%%%
%%% Known Facts  %%%
%%%%%%%%%%%%%%%%%%%%
\section{Facts}

In late 2007, Google (along with a handful of other companies) announced the
Open Handset Alliance \cite{open-handset-alliance-ann} along with the operating
system they wished to promote: Android.  The Android operating system uses a
modified Linux kernel, along with a slew of other open source projects, in
order to provide an open source operating system for phones.  Google provides a
software development kit (SDK) that allows developers to write applications for
Android in the Java programming language.

% Citations needed
Java is a programming language created by James Gosling that was originally
maintained by Sun Microsystems.  In 2006, Sun announced that Java's release
under the open source GNU GPL license. \cite{sun-open-sources-java} On August
20, 2009, the U.S.~Justice Department approved Oracle's deal to acquire Sun
Microsystems, along with its intellectual property: Java, MySQL, OpenOffice,
and many other notable projects. \cite{oracle-buys-sun}

In order to run Java code, one must use a virtual machine --- a program that
interprets bytecodes created by a Java compiler and translates it into
instructions the CPU can understand. \cite{javavm-bytecode} There are many
different implementations of the specification: HotSpot (Sun/Oracle's de facto
implemntation) \cite{hotspot}, Apache Harmony \cite{apache-harmony}, and
Red Hat's IcedTea \cite{icedtea}, to name a few.  The developers of Android
created a new virtual machine --- Dalvik.

Dalvik executes bytecodes in a custom executable format \cite{dalvik-bytecode}
that is different from Oracle's HotSpot virtual machine
\cite{javavm-bytecode}.  In order to fulfill the performance requirements and
restrictions imposed by a phone's hardware, Dalvik uses a different
architecture to execute code.  The Android SDK includes a program that converts
a Java class file produced by Oracle's Java compiler into the Dalvik executable
format. \cite{android-sdk-building}

On August 12, 2010, Oracle filed a lawsuit against Google, claiming that Dalvik
infringes upon their intellectual property. \cite{oracle-lawsuit}

%%%%%%%%%%%%%%%%%%%%%%%%%
%%% Research Question %%%
%%%%%%%%%%%%%%%%%%%%%%%%%
\section{Research Question}
Is Google's use of the Dalvik virtual machine in Android ethical?

%%%%%%%%%%%%%%%%%%%%%%%%%%%%%%%%%%%%%%%%%%%%%%
%%% Extant Arguments from External Sources %%%
%%%%%%%%%%%%%%%%%%%%%%%%%%%%%%%%%%%%%%%%%%%%%%
\section{Extant arguments}

Groklaw (a website run by Pamela Jones) posted this reply: \cite{groklaw}

\begin{quotation}
I expect Google would say Dalvik was an alternative to Java, not a version of
it. If indeed none of the Sun employees that ended up at Google worked on this,
and it's built from the ground up without any Sun technology or IP, on what
basis can Oracle prevail? Perhaps Oracle figures no room at Google is clean
enough. And of course clean room means nothing when it comes to patents.
\end{quotation}

Oracle's claim is that ``as a direct and proximate result of Google’s direct
and indirect willful copyright infringement, Oracle America has suffered, and
will continue to suffer, monetary loss to its business, reputation, and
goodwill.'' \cite[p.~9 line 11]{oracle-lawsuit}

%%%%%%%%%%%%%%%%%%%%%%%%%%%
%%% Analytic principles %%%
%%%%%%%%%%%%%%%%%%%%%%%%%%%
\section{Applicable analytic principles}

\begin{itemize}
    \item \cite[\S 2.02]{secode} Not knowingly use software that is obtained or
    retained either illegaly or unethically.

    \item \cite[\S 2.06]{secode} Identify, document, collect evidence and
    report to the client or the employer promptly if, in their opinion, a
    project is likely to fail, to prove too expensive, to violate intellectual
    property law, or otherwise to be problematic.

    \item \cite[\S 3.01]{secode} Strive for high quality, acceptable cost and a
    reasonable schedule, ensuring significant tradeoffs are clear to and
    accepted by the employer and the client, and are available for
    consideration by the user and the public.

    \item \cite[\S 6.04]{secode} Support, as members of a profession, other
    software engineers striving to follow this Code.

    \item \cite[\S 7.03]{secode} Credit fully the work of others and refrain
    from taking undue credit.
\end{itemize}

%%%%%%%%%%%%%%%%%%%%%%%%%%%%%%%%%%%%%%%
%%% Abstract your Expected Analysis %%%
%%%%%%%%%%%%%%%%%%%%%%%%%%%%%%%%%%%%%%%
\section{Abstract of Expected Analysis}

Google would be unethical in creating Dalvik if it violated the Software
Engineering Code of Ethics and Professional Practice \cite{secode} or if there
was a higher principle that it conflicts with.  Firstly, Google's
implementation of Dalvik does not use any of Oracle's HotSpot VM, it uses
portions of the Apache Harmony libraries (which are credited properly
\cite{dalvik-readme} and are preserved under Apache's license).  Secondly, by
creating a new virtual machine, Google was able to release Android under a
license that they deemed acceptable without illegally using Oracle's software,
thus respecting Section 2.02 of the Software Engineering Code \cite{secode}.

Furthermore, Google's creation of Dalvik upholds Section 3.01 of the Software
Engineering Code \cite{secode}, specifically, ``striv[ing] for high quality,
acceptable cost, and a reasonable schedule'' in a product.  Dalvik would not
have been completed in a reasonable amount of time without using portions of
Apache Harmony nor would it be as high of quality if it had used Oracle's
HotSpot.  There are other Java virtual machines that could have been used, but
none met the goals of the Android project.  The performance, power, and
licensing requirements that Google had for Android would not have been met by
an existing virtual machine, so creating Dalvik was the only ethical choice
that Google could be made.

\bibliographystyle{IEEEannot}

\bibliography{termpaper}
\end{document}

% vim: ft=tex tw=80

% Term paper proposal template - Ilona Sparks
% CSC 300: Professional Responsibilities
% Dr. Clark Turner

% One Column Format
\documentclass[11pt]{article}

\usepackage{multicol}
\usepackage{setspace}
\usepackage{url}

%%% PAGE DIMENSIONS
\usepackage{geometry} % to change the page dimensions
\geometry{letterpaper}


\begin{document}

\title{\vfill Coffee in Iceland: The Ethics of Java on Android} %\vfill gives us the black space at the top of the page
\author{
By Ross Light \vspace{10pt} \\
CSC 300: Professional Responsibilities  \vspace{10pt} \\
Dr. Clark Turner \vspace{10pt} \\
}
\date{\today} %Or use \Today for today's Date

\maketitle

\vfill  %in combination with \newpage this forces the abstract to the bottom of the page
\begin{abstract} % {{{1
Google's Android platform was released on November 5, 2007
\cite{open-handset-alliance-ann} as an open-source operating system for
consumer-level mobile devices, especially phones.  Android is based on Linux,
and exposes an application program interface (API) for the Java programming
language.  To improve performance, Android uses a custom virtual machine ---
called Dalvik --- to run the Java programs deployed on Android.  Three years
after Android's release, Oracle America Inc.~filed a lawsuit against Google
\cite{oracle-lawsuit}, claiming that Dalvik infringed on Oracle's registered
patents.  Google's use of the Dalvik virtual machine in Android is ethical
because it does not violate any portion of the Software Engineering Code of
Ethics and Professional Practice \cite{secode}.  Dalvik's creation was the only
way to meet performance, memory, and licensing restrictions, which aligns with
the Software Engineering Code's principle of building high quality software that
meets requirements.

(It should be noted that the author of this paper is employed by Google at
time of writing.)
\end{abstract} % }}}1

\thispagestyle{empty} %remove page number from title page, but still keep it as pg #1
\newpage

%Create a table of contents with all headings of level 3 and above.
%http://en.wikibooks.org/wiki/LaTeX/Document_Structure#Table_of_contents has
%info on customizing the table of contents
\thispagestyle{empty}  %Remove page number from TOC
\tableofcontents

\newpage

%start 2 column format
\begin{multicols}{2}
%Start numbering first page of content as page 1
\setcounter{page}{1}

\section{Facts} % {{{1
\label{sec:facts}

In late 2007, Google (along with a handful of other companies) announced the
Open Handset Alliance \cite{open-handset-alliance-ann} along with the operating
system they wished to promote: Android.  The Android operating system uses a
modified Linux kernel, along with a slew of other open source projects, in
order to provide an open source operating system for phones.  Google provides a
software development kit (SDK) that allows developers to write applications for
Android in the Java programming language.

Java is a programming language created by James Gosling that was originally
maintained by Sun Microsystems.  In 2006, Sun announced Java's release under the
open source GNU GPL license. \cite{sun-open-sources-java} On August 20, 2009,
the U.S.~Justice Department approved Oracle's deal to acquire Sun Microsystems,
along with its intellectual property: Java, MySQL, OpenOffice, and many other
notable projects. \cite{oracle-buys-sun}

In order to run Java code, one must use a virtual machine --- a program that
interprets bytecodes created by a Java compiler and translates it into
instructions the CPU can understand. \cite{javavm-bytecode} There are many
different implementations of the JavaVM specification: HotSpot (Sun/Oracle's de
facto implementation) \cite{hotspot}, Apache Harmony \cite{apache-harmony}, and
Red Hat's IcedTea \cite{icedtea}, to name a few.  The developers of Android
created a new virtual machine --- Dalvik.

Dalvik executes bytecodes in a custom executable format \cite{dalvik-bytecode}
that is different from Oracle's HotSpot virtual machine \cite{javavm-bytecode}.
In order to fulfill the performance requirements and restrictions imposed by a
phone's hardware, Dalvik uses a different architecture to execute code.  The
Android SDK includes a program that converts a Java class file produced by
Oracle's Java compiler into the Dalvik executable format.
\cite{android-sdk-building}  In order to provide compatibility with the Java
libraries, Dalvik includes portions of the Apache Harmony library.
\cite{apache-harmony} \cite{dalvik-readme}  The Apache Harmony project is
released under the Apache License 2.0 \cite{apache-license} which is ``preferred
by many companies because such licenses make it possible to use open-source
software code without having to turn proprietary enhancements back over to the
open source software community.'' \cite{why-apache2-license}

On August 12, 2010, Oracle filed a lawsuit against Google, claiming that Dalvik
infringes upon their intellectual property. \cite{oracle-lawsuit}

% end facts }}}1

\section{Research Question} % {{{1
\label{sec:question}
Is Google's use of the Dalvik virtual machine in Android ethical?

% end question }}}1

\section{Extant Arguments} % {{{1
\label{sec:args}

\subsection{Oracle's Claim} % {{{2
\label{sub:args_oracle}

Oracle's claim is that ``as a direct and proximate result of Google’s direct
and indirect willful copyright infringement, Oracle America has suffered, and
will continue to suffer, monetary loss to its business, reputation, and
goodwill.'' \cite[p.~9 line 11]{oracle-lawsuit}

% sub args_oracle }}}2

\subsection{Groklaw's Response} % {{{2
\label{sub:args_groklaw}

Groklaw (a website run by Pamela Jones) posted this reply: \cite{groklaw}

\begin{quotation}
I expect Google would say Dalvik was an alternative to Java, not a version of
it. If indeed none of the Sun employees that ended up at Google worked on this,
and it's built from the ground up without any Sun technology or IP, on what
basis can Oracle prevail? Perhaps Oracle figures no room at Google is clean
enough. And of course clean room means nothing when it comes to patents.
\end{quotation}

% sub args_groklaw }}}2

% end args }}}1

\section{Analysis} % {{{1
\label{sec:analysis}

%\begin{itemize}
%    \item \cite[\S 2.02]{secode} Not knowingly use software that is obtained or
%    retained either illegaly or unethically.
%
%    \item \cite[\S 2.06]{secode} Identify, document, collect evidence and
%    report to the client or the employer promptly if, in their opinion, a
%    project is likely to fail, to prove too expensive, to violate intellectual
%    property law, or otherwise to be problematic.
%
%    \item \cite[\S 3.01]{secode} Strive for high quality, acceptable cost and a
%    reasonable schedule, ensuring significant tradeoffs are clear to and
%    accepted by the employer and the client, and are available for
%    consideration by the user and the public.
%
%    \item \cite[\S 6.04]{secode} Support, as members of a profession, other
%    software engineers striving to follow this Code.
%
%    \item \cite[\S 7.03]{secode} Credit fully the work of others and refrain
%    from taking undue credit.
%\end{itemize}

\subsection{Criteria for Analysis} % {{{2
\label{sub:analysis_criteria}

In order to establish whether Google's actions are ethical or unethical, it is
important to provide an objective set of standards that are well-accepted for
the basis of this analysis.  At first glance, using Google's Code of Conduct
\cite{google-conduct} seems like a reasonable choice, and many parts of the Code
of Conduct are applicable here.  However, this analysis will use the ACM
Software Engineering Code of Ethics and Professional Practice \cite{secode} as
it is non-partial and includes more restrictions than Google's Code of Conduct
does.

By this measure, Google would be unethical in creating Dalvik if it violated the
Software Engineering Code of Ethics and Professional Practice \cite{secode} or
if there was a higher principle that it conflicts with.

% subsection analysis_criteria }}}2

\subsection{Licensing} % {{{2
\label{sub:analysis_licensing}

Firstly, Google's implementation of Dalvik does not use any of
Oracle's~HotSpot~VM, it uses portions of the Apache~Harmony libraries (which are
credited properly \cite{dalvik-readme} and are preserved under Apache's
license).  The Apache~Harmony is released under Apache~License~2.0 --- a
software license that grants rights to redistribute software, as long as
distributors preserve the copyright notices and note any modifications made.
\cite{apache-license} Google is then obligated to distribute Dalvik under a
license that has terms compatible with the Apache~License~2.0.  Google
distributes Dalvik under the same Apache~License~2.0 and credits Harmony in the
Dalvik README.  \cite{dalvik-readme}

Secondly, by creating a new virtual machine, Google was able to release Android
under a license that they deemed acceptable without illegally using Oracle's
software, thus respecting Section 2.02 of the Software Engineering Code.
\cite[\S 2.02]{secode}  One of the requirements of Android was to provide a
common platform for a wide variety of phones \cite{open-handset-alliance-ann},
but the only way to ensure a common platform is to encourage widespread
adoption.  Companies prefer the Apache License 2.0 because of its less
restrictive nature than the GPL. \cite{why-apache2-license}

% subsection analysis_licensing }}}2

\subsection{Quality} % {{{2
\label{sub:analysis_quality}

Furthermore, Google's creation of Dalvik upholds Section 3.01 of the Software
Engineering Code \cite{secode}, specifically, ``striv[ing] for high quality,
acceptable cost, and a reasonable schedule'' in a product.  Dalvik would not
have been completed in a reasonable amount of time without using portions of
Apache Harmony nor would it be as high quality if it had used Oracle's HotSpot.
HotSpot doesn't fit within the performance characteristics of lower-end phones.
There are other Java virtual machines that could have been used, but none met
the goals of the Android project.

% subsection analysis_quality }}}2

% section analysis }}}1

\section{Conclusion} % {{{1
\label{sec:conclusion}

The performance, power, and licensing requirements that Google had for Android
would not have been met by an existing virtual machine, so creating Dalvik was
the only ethical choice that Google could be made. 

% section conclusion }}}1

\end{multicols}
\newpage

%cite all the references from the bibtex you haven't explicitly cited
\nocite{*}
\bibliographystyle{IEEEannot}
\bibliography{termpaper}

\end{document}

% vim: ft=tex tw=80 fdm=marker fdc=3 spell

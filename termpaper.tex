% Term paper proposal template - Ilona Sparks
% CSC 300: Professional Responsibilities
% Dr. Clark Turner

% One Column Format
\documentclass[11pt]{article}

\usepackage{multicol}
\usepackage{setspace}
\usepackage{url}

%%% PAGE DIMENSIONS
\usepackage{geometry} % to change the page dimensions
\geometry{letterpaper}


\begin{document}

\title{\vfill Coffee in Iceland: The Ethics of Java on Android} %\vfill gives us the black space at the top of the page
\author{
By Ross Light \vspace{10pt} \\
CSC 300: Professional Responsibilities  \vspace{10pt} \\
Dr. Clark Turner \vspace{10pt} \\
}
\date{\today} %Or use \Today for today's Date

\maketitle

\vfill  %in combination with \newpage this forces the abstract to the bottom of the page
\begin{abstract} % {{{1

Google released Android on November 5, 2007 \cite{open-handset-alliance-ann} as
a platform for mobile phones. Programmers can write applications for Android in
Java.  These programs are run with Dalvik --- a virtual machine that Google
created.  In 2010, Oracle~America~Inc.\ filed a lawsuit against Google
\cite{oracle-lawsuit}, claiming that Dalvik infringed on Oracle's intellectual
property.  This raises the question, is Google's use of the Dalvik virtual
machine in Android ethical?  Creating Dalvik was the only way to ethically meet
performance, memory, and licensing requirements, which aligns with the spirit of
the ACM Software Engineering Code of Ethics and Professional Practice
\cite{secode}.

(It should be noted that the author of this paper is employed by Google at time
of writing.)

\end{abstract} % }}}1

\thispagestyle{empty} %remove page number from title page, but still keep it as pg #1
\newpage

%Create a table of contents with all headings of level 3 and above.
\thispagestyle{empty}  %Remove page number from TOC
\tableofcontents

\newpage

%start 2 column format
\begin{multicols}{2}
%Start numbering first page of content as page 1
\setcounter{page}{1}

\section{Definitions} % {{{1
\label{sec:defs}

\begin{description}
    \item[Virtual Machine] A program that runs instructions that are not
    natively understood by a CPU.
    \item[Bytecode] An instruction understood by a virtual machine.
    \item[JavaVM] The set of bytecodes that is documented by ``The Java Virtual
    Machine Specification''. \cite{javavm-bytecode}
    \item[SDK] Software Development Kit.  An SDK is a set of programs and other
    materials that allows programmers to write programs for a particular
    environment.
    \item[Open Source] If a software project released as open source, it means
    that the software's source code can be obtained and reused legally.  There
    are many licenses that are considered open source, each with different
    terms.
    \item[Apache License 2.0] An open source license that allows modification
    and reuse of the source code, even if the user don't redistribute the
    modified source code. \cite{apache-license}
    \item[GNU GPL License] GNU General Public License.  An open source license
    that allows modification and reuse of the source code, as long as users
    distribute the source code of their modifications. \cite{gpl-license}
    \item[Library] A collection of code that is useful in many different
    programs (e.g.\ text handling, graphics, etc.).
\end{description}

% section defs }}}1

\section{Facts} % {{{1
\label{sec:facts}

In late 2007, Google (along with a handful of other companies) announced the
Open Handset Alliance \cite{open-handset-alliance-ann} along with the operating
system they wished to promote: Android.  The Android operating system is built
entirely from open source projects.  Most of the projects Android uses are
released under the Apache License 2.0, with the exception of the Linux kernel
(which is released under the GNU GPL).  Google provides a software development
kit (SDK) that allows developers to write applications for Android in the Java
programming language.

Java is a programming language created by James Gosling that was originally
maintained by Sun Microsystems.  In 2006, Sun announced Java's release under the
open source GNU GPL license. \cite{sun-open-sources-java} On August 20, 2009,
the U.S.~Justice Department approved Oracle's deal to acquire Sun Microsystems
and its intellectual property, including Java. \cite{oracle-buys-sun}

Many programming languages are converted from text directly into instructions
that the CPU understands.  Java does not get converted directly to CPU
instructions --- in order to run Java code, one must use a virtual machine.  A
virtual machine guarantees that the same program will run on any CPU without
modification.  Once a virtual machine has been written for a particular CPU
architecture, it can execute programs written for the virtual machine.

The Java compiler produces a documented, well-defined set of standard bytecodes,
known as the JavaVM specification. \cite[Chapter 4]{javavm-bytecode}  There are
many different implementations of the JavaVM specification: HotSpot
(Sun/Oracle's de facto implementation) \cite{hotspot}, Apache Harmony
\cite{apache-harmony}, and Red Hat's IcedTea \cite{icedtea}, to name a few.

The developers of Android created a new virtual machine --- Dalvik.  Dalvik
executes bytecodes in a format that is different from JavaVM in order to fulfill
the performance requirements on mobile phones.  \cite{dalvik-bytecode}
\cite{dalvik-vm-arch} The Android~SDK includes a program, dex, that converts a
Java bytecodes produced by Oracle's Java compiler into Dalvik bytecodes.
\cite{android-sdk-building} In order to provide compatibility with the Java
libraries, Dalvik includes portions of the Apache~Harmony library.
\cite{apache-harmony} \cite{dalvik-readme}  Apache~Harmony and Dalvik are
released under the Apache License 2.0 \cite{apache-license} which is ``preferred
by many companies because such licenses make it possible to use open-source
software code without having to turn proprietary enhancements back over to the
open source software community.'' \cite{why-apache2-license}

On August 12, 2010, Oracle filed a lawsuit against Google, claiming that Dalvik
infringes upon their intellectual property. \cite[Count VIII]{oracle-lawsuit}

% end facts }}}1

\section{Research Question} % {{{1
\label{sec:question}
Is Google's use of the Dalvik virtual machine in Android ethical?

% end question }}}1

\section{Extant Arguments} % {{{1
\label{sec:args}

\subsection{Oracle's Claim} % {{{2
\label{sub:args_oracle}

Oracle's claim is that ``as a direct and proximate result of Google's direct
and indirect willful copyright infringement, Oracle America has suffered, and
will continue to suffer, monetary loss to its business, reputation, and
goodwill.'' \cite[p.~9 line 11]{oracle-lawsuit}

The first claim being made is that Google is infringing on Oracle's copyright.
The second claim is that by infringing, Google is harming the public by causing
losses to Oracle's employees and investors.  Either of these actions would be
unethical, and the first would certainly be illegal.

% TODO: Discuss specific patents: JIT, Access to Resources, etc.

% sub args_oracle }}}2

\subsection{Groklaw's Response} % {{{2
\label{sub:args_groklaw}

When the lawsuit was announced, the website Groklaw posted an analysis of the
lawsuit.  Groklaw is a website run by Pamela Jones, an open source advocate who
previously worked as a paralegal. \cite{groklaw-pj} Jones responded to Oracle's
lawsuit, saying: \cite{groklaw}

\begin{quotation}
I expect Google would say Dalvik was an alternative to Java, not a version of
it. If indeed none of the Sun employees that ended up at Google worked on this,
and it's built from the ground up without any Sun technology or [Intellectual
Property], on what basis can Oracle prevail? Perhaps Oracle figures no room at
Google is clean enough. And of course clean room means nothing when it comes to
patents.
\end{quotation}

Jones's argument is that if Google has not used any of Oracle's software to
build Dalvik, then there's no violation of intellectual property, and thus no
unethical behavior.

% sub args_groklaw }}}2

% end args }}}1

\section{Analysis} % {{{1
\label{sec:analysis}

%\begin{itemize}
%    \item \cite[\S 2.02]{secode} Not knowingly use software that is obtained or
%    retained either illegaly or unethically.
%
%    \item \cite[\S 2.06]{secode} Identify, document, collect evidence and
%    report to the client or the employer promptly if, in their opinion, a
%    project is likely to fail, to prove too expensive, to violate intellectual
%    property law, or otherwise to be problematic.
%
%    \item \cite[\S 3.01]{secode} Strive for high quality, acceptable cost and a
%    reasonable schedule, ensuring significant tradeoffs are clear to and
%    accepted by the employer and the client, and are available for
%    consideration by the user and the public.
%
%    \item \cite[\S 6.04]{secode} Support, as members of a profession, other
%    software engineers striving to follow this Code.
%
%    \item \cite[\S 7.03]{secode} Credit fully the work of others and refrain
%    from taking undue credit.
%\end{itemize}

\subsection{Criteria for Analysis} % {{{2
\label{sub:criteria}

In order to establish whether Google's actions are ethical or unethical, it is
important to provide an objective set of standards that are well-accepted for
the basis of this analysis.  At first glance, using Google's Code of Conduct
\cite{google-conduct} seems like a reasonable choice, and many parts of Google's
Code of Conduct are applicable here.  However, this analysis will use the ACM
Software Engineering Code of Ethics and Professional Practice \cite{secode}, as
it is non-partial and includes more restrictions than Google's Code of Conduct
does.

By this measure, Google would be unethical in creating Dalvik if it violated the
Software Engineering Code of Ethics and Professional Practice \cite{secode} or
if there was a higher principle that it conflicts with.

% subsection criteria }}}2

\subsection{Requirements} % {{{2
\label{sub:requirements}

As part of a software engineer's duty to the public, a software engineer should
``approve software only if they have a well-founded belief that it is safe,
meets specifications, passes appropriate tests, and does not diminish quality of
life, diminish privacy or harm the environment.'' \cite[\S~1.03]{secode} If
Dalvik does not meet its requirements, then it is unethical.

Android's stated goal is to ``give consumers a far better user experience than
much of what is available on today's mobile platforms'' and ``provide developers
a new level of openness that enables them to work more collaboratively'' on
``thousands of different phone models.'' \cite{open-handset-alliance-ann}
Android must be open source, it must work similarly on a variety of phones, and
must provide an acceptable user experience.

\subsubsection{Licensing} % {{{3
\label{sub:licensing}

One of the requirements of Android was to provide a common platform for a wide
variety of phones \cite{open-handset-alliance-ann}; the only way to ensure a
common platform is to encourage widespread adoption.  Companies prefer the
Apache License 2.0 because of its less restrictive nature than the GPL.
\cite{why-apache2-license}  Thus, to promote adoption, Google released Android
under the Apache License 2.0.

Google's implementation of Dalvik does not use any of Oracle's~HotSpot~VM, but
it uses portions of the Apache~Harmony libraries.  In order for this to be
ethical, the Software Engineering Code dictates that software engineers ought to
``credit fully the work of others and refrain from taking undue credit.''
\cite[\S 7.03]{secode}  Google distributes Dalvik under the same
Apache~License~2.0 and credits Harmony in the Dalvik README.
\cite{dalvik-readme}  The principle of giving credit is upheld.

Section 2.02 of the Software Engineering Code states that software engineers
should ``not knowingly use software that is obtained or retained either
illegally or unethically.'' \cite[\S 2.02]{secode}  Instead of using Oracle's
libraries, Google used the Apache Harmony libraries.  Dalvik uses its own
executable format that is distinct from JavaVM.  Google created Dalvik and then
used Apache~Harmony to supplement its functionality, obeying all the terms of
the Apache~License~2.0.  All software was either created at Google or obtained
legally.

% subsubsection licensing }}}3

\subsubsection{Why a Virtual Machine?} % {{{3
\label{ssub:why-vm}

(TODO: A virtual machine is necessary to provide portability and a consistent
user experience.)

% subsubsection why-vm }}}3

\subsubsection{User Experience} % {{{3
\label{ssub:quality}

The Software Engineering Code \cite{secode} encourages software engineers to
``strive for high quality, acceptable cost, and a reasonable schedule'' in a
product.  Dalvik would not have been completed in a reasonable amount of time
without using portions of Apache Harmony, nor would it be as high quality if it
had used Oracle's HotSpot.  HotSpot doesn't fit within the performance
characteristics of lower-end phones. \cite{dalvik-vm-arch} There are other Java
virtual machines that could have been used, but none met the goals of the
Android project.

% subsubsection quality }}}3

% subsection requirements }}}2

\subsection{The Use of Java} % {{{2
\label{sub:java}

\subsubsection{The dex Program} % {{{3
\label{ssub:dex}

dex, a part of the Android SDK, is a separate program that converts Java
bytecodes to Dalvik bytecodes.  Converters could be written that convert other
programming languages to the Dalvik executable format; there is no dependency on
Java.  Dalvik includes libraries that are similar to those in the Java standard
library, but creating an API that's similar to another is not an unethical act,
as long as the underlying implementations differ.

% subsubsection dex }}}3

\subsubsection{Damages to Oracle} % {{{3
\label{ssub:oracle-damage}

(TODO: I will address the claim that damage has been done to Oracle.  Oracle has
not been damaged, and that by promoting the use of Java, Google's is actually
encouraging use of Oracle's products.)

% subsubsection oracle-damage }}}3

% subsection java }}}2

% section analysis }}}1

\section{Conclusion} % {{{1
\label{sec:conclusion}

The performance, power, and licensing requirements that Google had for Android
would not have been met by an existing virtual machine, so creating Dalvik was
the only ethical choice that Google could make.

% section conclusion }}}1

\end{multicols}
\newpage

%cite all the references from the bibtex you haven't explicitly cited
\nocite{*}
\bibliographystyle{IEEEannot}
\bibliography{termpaper}

\end{document}

% vim: ft=tex tw=80 fdm=marker fdc=3 spell
